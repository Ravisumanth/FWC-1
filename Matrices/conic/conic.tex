\documentclass[a4paper,12pt,twocolumn]{article}
\usepackage{graphicx}
\usepackage[margin=0.5in]{geometry}
\usepackage[cmex10]{amsmath}
\usepackage{array}
\usepackage{gensymb}
\usepackage{booktabs}
\title{Conic Assignment}

\author{Ravi Sumanth Muppana- FWC22003}
\date{September 2022}
\providecommand{\norm}[1]{\left\lVert#1\right\rVert}
\providecommand{\abs}[1]{\left\vert#1\right\vert}
\let\vec\mathbf
\newcommand{\myvec}[1]{\ensuremath{\begin{pmatrix}#1\end{pmatrix}}}
\newcommand{\mydet}[1]{\ensuremath{\begin{vmatrix}#1\end{vmatrix}}}
\providecommand{\brak}[1]{\ensuremath{\left((#1\right)}}
\begin{document}
\maketitle
\section{Problem:}
Find the equation of circle passing with radius $5$ whose center lies on x-axis and passes through point $\myvec{2,3}$.
\maketitle
\section{Solution:}
\begin{figure}[h]
	\includegraphics[width=\linewidth]{conic.png}
\caption{Circle}
\end{figure}
\subsection{Theory:}
The circle equation when it's center and radius are given is
\begin{align}
	&\vec{(x-a)^2} + \vec{(y-b)^2} = \vec{r^2}\\
\end{align}
where the centre of the circle is $\myvec{a\\b}$.
\subsection{Mathematical Calculation:}
Given the radius of circle is $5$. The circle passes through a point $\myvec{2\\3}$. Also, the center of circle is assumed as $\myvec{a\\0}$. Substitute $\myvec{a\\0}$ in eq.$1$ we get,
\begin{align}
	&\vec{(x-a)^2} + \vec{(y)^2} = \vec{25}\\
\end{align}
As the point $\myvec{2\\3}$ passes through the circle, substitute $\myvec{2\\3}$ in the equation, we get,
\begin{align}
	&\vec{(2-a)^2} + \vec{(3)^2} = \vec{25}\\
	&\vec{4+a^2-2a} + \vec{9} = \vec{25}\\
	&\vec{a^2-2a+13} = \vec{25}\\
	&\vec{a^2-2a-12} = 0\\
\end{align}
The roots of the equation will be $(6,-2)$. Hence, the center of the circle can be $\myvec{6\\0}$ or $\myvec{-2\\0}$.
The equation of circle will therefore be,
\begin{align}
	&\vec{(x-6)^2} + \vec{y^2} = 25\\
	&\vec{(x+2)^2} + \vec{y^2} = 25
\end{align}
\section{Construction:}

\begin{table}[h]
        \centering
\setlength\extrarowheight{2pt}
        \begin{tabular}{|c|c|c|}
                \hline
                \textbf{variable} & \textbf{length/point} & \textbf{Description}\\
                \hline
		A & np.roots(coeff) & coeff = (1,-4,-12)\\
		\hline
		c & $(a-A[0])^2+b^2-r^2$ & Circle Eqn\\
		\hline
        \end{tabular}
\end{table}
\end{document}
